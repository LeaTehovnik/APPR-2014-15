\documentclass[11pt,a4paper]{article}

\usepackage[slovene]{babel}
\usepackage[utf8x]{inputenc}
 \usepackage[unicode]{hyperref}
\usepackage{graphicx}
\usepackage{pdfpages}
\usepackage{hyperref}
\usepackage{float}
\usepackage[font=small,skip=0pt]{caption}

\pagestyle{plain}

\begin{document}
\title{Poročilo pri predmetu \\
Analiza podatkov s programom R\\
\vspace{3mm}
\textbf{{Rekord maratona skozi čas}}}
\author{Lea Tehovnik}
\maketitle


\section{Izbira teme}

Ker del svojega prostega časa namenim tudi teku, sem si za temo mojega projekta izbrala rekord maratona skozi čas.  Ukvarjala se bom z obdelavo podatkov različnih rekordov maratona. V projektu bom za vsak rekord podala:
\begin{enumerate}
\item{\verb|Ime maratonca| (imenska spremenljivka),}
\item{\verb|Državljanstvo| (imenska spremenljivka),}
\item{\verb|Kraj oz. ime maratona| (imenska spremenljivka),}
\item{\verb|Datum maratona| (številska spremenljivka),}
\item{\verb|Čas rekorda| (številska spremenljivka),}
\item{\verb|Spol tekmovalca| (urejenostna spremenljivka),}
\item{\verb|Država maratona|(imenska spremenljivka).}
\end{enumerate}
 

Podatke sem pridobila iz spletne strani Wikipedia, natačneje na naslovu \url{http://en.wikipedia.org/wiki/Marathon_world_record_progression}.

Moj cilj je analizirati različne podatke, npr. ali je ista oseba postavila rekord večkrat, kje je bil največkrat postavljen, kdo je večkrat podrl rekord, moški ali ženske, ter iz katere države je največ rekorderjev. Če pa bom tekom projekta dobila še kakšno idejo, bom obdelala tudi to.

Glede na izkušnje predvidevam, da bodo rekord večkrat podrli moški, saj veljajo za fizično močnejše (kar pa seveda ni vedno res). Druga hipoteza je, da bo rekord verjetno največkrat podrt v državah, katere maratoni so svetovno znani, npr. maraton skozi New York, na katerem je tekel tudi marsikateri Slovenec in ima skozi leta vse večjo udeležbo.

\pagebreak
\section{Obdelava, uvoz in čiščenje podatkov}

Podatke sem s pomočjo datoteke xml.r uvozila, ker sem imela dve tabeli z istimi spremenljivkami sem ju za lažjo obdelavo podatkov kar združila skupaj in jih uredila v preglednico maraton. Ker sta bila zadnja dva stolpca brezpomenska, sem ju izbrisala, dodala pa sem nov stolpec, ki bo določal spol tekomovalca. Prav tako sem s pomočjo funckije casvsekunde čas, ki je bil podan v urah in minutah, ter ločen s podpičji, spremenila v čas, ki je določen samo s sekundami, kar mi bo gotovo koristilo pri nadaljni obravnavi podatkov v sledečih fazah.

\pagebreak
Želela sem izvedeti iz katerih držav prihaja največ maratoncev. Pri tem sem si pomagala z grafom in ugotovila, da jih največ prihaja iz Združenih držav Amerike ter Velike Britanije, kar je razmeroma logično, saj gre za razviti državi z veliko populacije, prav tako so ravno v teh državah najbolj znani maraton(Boston, New York, London), kar verjetno še bolj spodbudi državljane k športnemu udejstvovanju.

\begin{figure}[H]
  \includegraphics[width=\textwidth]{../slike/graf1.pdf}
  \caption{Državljanstva maratoncev}
  \label{fig:Slika 1}
\end{figure}


Drugi graf se nanaša na razmerje med spoloma. Očitno so rekord večkrat podrli moški, kar pa tudi potrdi našo hipotezo. Znano je namreč, da se žensko in moško telo precej razlikujeta, moško telo ima načeloma več mišič in sposobnost, da le te gradi hitreje.
Razlog, da so rekord večkrat podrli moški, se skriva tudi v dejstvu, da je bil maraton včasih rezerviran predvsem za moške. Violet Piercy iz Velike Britanije je bil prva ženska, katere čas maratona so uradno izmerili na Britanski tekmi, 3. oktobra leta 1926. Zaradi nizke konkurence med ženskamim, je bil ta rekord nepremagljiv kar 37 let. Da se je to končalo, je kriva Američanka Merry Lepper, ki je rekord Piercijeve popravila za dobre tri minute.
Izključevanje žensk iz maratonov se je končalo leta 1967, ko je Kathrine Switzer na Bostonskem maratonu s svojim fantom uspešno pretekla 42 kilometov, kljub temu, da so jo med tekom nekateri moški zaničevali in vlekli za majico. Od takrat dalje pa se je stanje ženskih rekordov počasi začelo izboljševati in ženski rekordi so se postavljali sorazmerno z moškimi.



\begin{figure}[H]
  \includegraphics[width=\textwidth]{../slike/graf2.pdf}
  \caption{Razmerje med spoloma}
  \label{fig:Slika 2}
\end{figure}

\newpage
\section{Analiza in vizualizacija podatkov}

V tretji fazi je bila naša naloga, da dva podatka, imenskega ali urejenostnega ter številskega podamo na zemljevidu. 
Odločila sem se, da bom preučila kolikokrat je bil na posameznem maratonu postavljen rekord in to na zemljevidu predstavila kot številko spremenjlivko - jakost barve bo določala število, močnejša bo barva, večkrat je bil podrt rekord in obratno. Za ta namen je bilo potrebno podatke malo preurediti. Iz stolpca Kraj sem zato izluščila države maratonov. 
Razberemo lahko, da je bil rekord maratona največkrat postavljen v Združenih Državah Amerike, nato v Veliki Britaniji, Nemčiji, Japonski, na Nizozemskem in tako naprej. Razlog zato je morda tudi priljubljenost teh maratonov, saj maratoni v Londonu, Bostonu, New Yorku, Berlinu in Parizu veljajo za ene izmed najbolj zanimivih na svetu in imajo posledično tudi veliko številko tekmovalcev, kar pa še poveča možnost za nov rekord. Prav tako menim, da bo imel maraton v Bostonu zaradi tragičnih dogokov leta 2013 skozi leta še večji obisk, ker nekaterim udejstvovanje pomeni veliko več, kot le preteči 42 kilometov.

\begin{figure}[H]
  \makebox[\textwidth][c]{
  \includegraphics[width=1.2\textwidth]{../slike/maraton_svet1.pdf}
  }
  \caption{Število postavljenih rekordov}
  \label{fig:Zemljevid 1}
\end{figure}

Za imenski ali urejenostni podatek pa sem analizirala ali je rekord maratona večkrat postavil državljan države, kjer se je odvil dogodek, tujec, ali sta ga oba postavila enakokrat. Torej, če je državljan večkrat postavil rekord, bo država, kjer se je odvil maraton, obarvana z modro, če tujec z rdečo in nazadnje z zeleno, če sta rekord oba postavila enakokrat.
Ker sem pri številski spremenljivki že dobila države maratonov, je bilo ugotavljanje, ali se država ujema z državljanstvom, lažje.
Nazadnje sem na zemjevid dodala še oznake. Na zemljevidu sem označila nekatera glavna mesta, kjer so se odvijali maratoni.
Razberemo lahko, da so V Združenih Državah Amerike, Veliki Britaniji, Franciji in na Japonskem rekord večkrat postavili državljani, kar je za ZDA, Francijo in Veliko Britanijo kar presentljivo, saj gre za države z veliko priseljenci in turisti, ki se maratona prav načrtno udeležijo. 

\begin{figure}[H]
  \makebox[\textwidth][c]{
  \includegraphics[width=1.2\textwidth]{../slike/maraton_svet2.pdf}
  }
  \caption{Ujemanje državljanstva z državo}
  \label{fig:Zemljevid 2}
\end{figure}

\newpage
\section{Napredna analiza podatkov}

Za četrto fazo sem se odločila iz podatkov razbrati še nekaj informacij. Najprej sem se odločila, da narišem graf, ki prikazuje spreminjanje rekorda maratona skozi čas, kar je pravzaprav tudi naslov mojega projetka. Ker sem tekom projekta pridobila nekaj izkušenj, je bilo risanje tega grafa precej lažje kot risanje grafov v drugi fazi.

Ker se rekordi glede na spol razlikujejo, sem obravnala vsak spol zase.

Pa si poglejmo najprej graf rekordov, ki so jih postavile ženske.
Iz grafa lahko razberemo, da so v postavljanju rekordov tri časovne luknje. Menim, da je vzrok tega predvsem družbeno in gospodarsko stanje. Prvi upad rekordov se pokaže po letu 1918, v času po 2. svetovni vojni, ko so bile razmere po svetu še vedno zelo zaostrene in so se ljudje predvsem osredotočali na boj za preživetje kot udejstvovanje na maratonih.
Največji razlog za upad rekordov med letom 1918 in letom 1963 pa je prav gotovo že zgoraj omenjeno izključevanje žensk iz maratona.
Prav tako rekordov ni v času druge svetovne vojne, tudi zaradi zgoraj naštetih razlogov. Kar je za čas krize in vojn povsem normalno, saj so bile leta 1940 zaradi vojne odpovedane tudi olimpijske igre na Japonskem.
Sledilo je obdobje hladne vojne, leta 1961 so postavili Berlinski zid, zato ne preseneča, da je bil naslednji rekord v Ameriki, kjer je bila situacija sicer gospodarsko oslabljena, a bolj umirjena. Tudi sicer so bili rekordi postavljeni v državah, kjer ni bilo toliko napestosti, predvsem v Veliki Britaniji in Združenih državah Amerike.
Od sredine osemdesetih pa do konca hladne vojne(tj. 1990 z razapadom Sovjetske zveze) je bilo spet obdobje, ko ni bil postavljen noben rekord, predvidevam, da delno tudi zaradi gospodarskih in družebnih razlogov, spomnimo se npr., da je bil leta 1989 podrt Berlinski zid.
V današnji dobi je bil zadnji rekord podrt leta 2003. To je uspelo Pauli Radcliffe in sicer že drugo leto zapored.

Z uporabo regresijske analize sem želela še izvedeti, kaj naj bi se z rekordom dogajalo v prihodnje. Uporabila sem različne načine prileganja, izključila sem tiste, katerih vrednosti so se povečevale, saj vemo ,da bo rekord vedno manjši.
Uporabila sem tri metode : linearen model, eksponentni model in loess model. Poiskusila sem tudi z ostalimi, a so napovedi in prileganja preveč odstopala od realnosti.
Pri lm modelu sem poiskusila kar nekaj enačb z različnimi eksponenti, a sta na koncu najbolj zadostovali dve :
\begin{enumerate}
\item{\verb|lm(cas ~ datt)| (linearna, čas je linearno odvisen z datumom),}
\item{\verb|lm(logcas ~ datt)| (eksponentna, logaritem časa je linearno odvisen z datumom),}

\end{enumerate}

Kot pričakovano nam vsi modeli napovejo, da se bo čas rekorda z leti zniževal, sama se nastavila tud ylim, saj verjetno nihče ne bo pretekel rekorda v manj kot 5000 sekundah.

Predvidevam, da bodo razlike v rekordih z leti vse manjše, saj je potrebna res ogromna fizična vzdržljivost, da  pretečeš 42 kilometrov v tako malo časa. Človeško telo lahko treniramo do neke meje, a mislim, da bo rekord vse težje postaviti, saj je čas že sedaj nastavljen zelo nizko.

\begin{figure}[H]
  \makebox[\textwidth][c]{
  \includegraphics[width=1.2\textwidth]{../slike/graf4.pdf}
  }
  \caption{Spreminjanje moškega rekorda s časom}
  \label{fig:Slika 4}
\end{figure}

Pri moški je sitaucija malo bolj raznolika, predvsem ni tako velikih prepadov kot pri ženskah, za kar je kriva tudi neenakopravnost med spoloma v začetku 20. stoletja. Kljub temu lahko prav tako zaznamo tri in sicer v obdobju med in po 1. svetovni vojni, v obdobju pred in po 2. svetovno vojno in tik pred koncem hladne vojne. Razlogi so podobni kot pri ženskah, saj gre za isto časovno obodbje, družbo, politično situacijo , zato tudi ne preseneča, da sta si grafa vizualno dokaj podobna. Opazimo edino, da moški tudi v današnji dobi konstantno podirajo rekorde, zadnji je bil namreč podrt kar leta 2014, sicer pa je bil rekrod zadnjih 6 let podrt prav v Berlinu, kar se v zgodovini sicer ni zgodilo velikrat, verjetno zaradi delitve Nemške države.
Tudi tu sem uporabila različne modele regresijske analize, poleg linearnega, eksponentnega in loess sem uporabila tudi Gam model. Ker sta loess in Gam kompleksnejša modela, podajam samo enačbi lm modela
\begin{enumerate}
\item{\verb|lm(cas.zenske ~ datt.zenske)| (linearna, čas je linearno odvisen z datumom),}
\item{\verb|lm(logcas.zenske ~ datt.zenske)| (eksponentna, logaritem časa je linearno odvisen z datumom).}
\end{enumerate}
Vidimo, da so razlike v maratonih v zadnjih 20 letih precej majhne, vzrok za to tiči(kot se omenila že zgoraj) predvsem v tem, da človeško telo lahko doseže le neko mejo fizične pripravljenosti, več kot to pa zelo težko. Najboljšo napoved nam da model Gam, ki napove, da se bo rekord postopoma zniževal ( tud itu se nastavila limito y osi). Menim, da je ta model še najbolj realen, saj ostali predvidijo kar velik padec rekordov, kar pa je, če gledamo iz telesnih sposobnosti človeka, zelo težko.


\begin{figure}[H]
  \makebox[\textwidth][c]{
  \includegraphics[width=1.2\textwidth]{../slike/graf5.pdf}
  }
  \caption{Spreminjanje ženskega rekorda s časom}
  \label{fig:Slika 5}
\end{figure}

\section{Zaključek}
Vidimo lahko, da na rekorde pomembno vplivajo tudi družbeni in gospodarski dejavniki ter, da je postavljanje le teh z leti čedalje težji in večji podvig.

\newpage
\begin{thebibliography}{4}
\bibitem{1}
  \url{http://en.wikipedia.org/wiki/Marathon_world_record_progression}\\
  {Accessed: 25-02-2015}
\bibitem{2}
  \url{http://en.wikipedia.org/wiki/20th_century#Wars_and_politics}\\
  {Accessed: 25-02-2015}
\bibitem{3}
  \url{http://en.wikipedia.org/wiki/Cold_War}\\
  {Accessed: 25-02-2015}
\bibitem{4}
  \url{http://www.marathonguide.com/history/olympicmarathons/chapter25.cfm}\\
  {Accessed: 25-02-2015}
  \bibitem{5}
  \url{ https://www.boundless.com/economics/textbooks/boundless-economics-textbook/immigration-economics-38/introduction-to-immigration-138/dimensionalizing-immigration-numbers-of-immigrants-around-the-world-544-12641/}\\
  {Accessed: 25-02-2015}
 

\end{thebibliography}

\end{document}
