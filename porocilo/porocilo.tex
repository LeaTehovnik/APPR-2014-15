\documentclass[11pt,a4paper]{article}

\usepackage[slovene]{babel}
\usepackage[utf8x]{inputenc}
\usepackage{graphicx}

\pagestyle{plain}

\begin{document}
\title{Poročilo pri predmetu \\
Analiza podatkov s programom R}
\author{Študent FMF}
\maketitle

\section{Izbira teme}

Tema mojega projekta je rekord maratona skozi čas. Ukvarjala se bom z obdelavo podatkov različnih rekordov maratona. V projektu bom za vsak rekord podala spol(urejenostna spremenljivka), čas(številska spremenljivka), ime maratonca, državljanstvo ter kraj oz. ime in datum maratona(vse imenske spremenljivke).

Podatke bom pridobila iz spletne strani Wikipedia, natačneje na naslovu http://en.wikipedia.org/wiki/Marathon_world_record_progression.

Moj cilj je analizirati različne podatke, npr. ali je ista oseba postavila rekord večkrat, kje je bil največkrat postavljen, kdo je večkrat podrl rekord, moški ali ženske ter iz katere države je največ rekorderjev. Če pa bom tekom projetka dobila še kakšno idejo, bom obedelala tudi to.

\section{Obdelava, uvoz in čiščenje podatkov}

Uvozila sem datoteke HTML z Wikipedie. Zaenkrat mi to še povrzroča težave, a upam, da bom tekom tedna našla rešitev...

\section{Analiza in vizualizacija podatkov}

\includegraphics{../slike/povprecna_druzina.pdf}

\section{Napredna analiza podatkov}

\includegraphics{../slike/naselja.pdf}

\end{document}
